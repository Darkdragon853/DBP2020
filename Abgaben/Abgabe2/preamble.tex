\documentclass[11pt,titlepage]{article}
\usepackage[utf8]{inputenc}
\usepackage{amsmath}
\usepackage{amssymb}
\usepackage[ngerman]{babel}
\usepackage{graphicx}
\usepackage{comment}
\usepackage{appendix}
\usepackage{caption, subcaption}
\usepackage[T1]{fontenc}
\usepackage{multirow}

\usepackage{pdfpages}
% für Unterstreichungen
\usepackage[normalem]{ulem}
%\usepackage{environ}
\usepackage{fancyhdr}
% \usepackage[
%   sorting=none
%   %backend=biber,
%   %style=numeric,
% ]{biblatex}

\usepackage[
  a4paper,
  includeheadfoot,
  left=20mm,
  right=20mm,
  top=20mm,
  bottom=20mm
]{geometry}

\usepackage{csquotes}
\usepackage{wrapfig}
\usepackage{longtable}
\usepackage{siunitx}

\usepackage{tabularx}
\usepackage{hyperref}
\usepackage{setspace}

\pagestyle{fancy}
\setlength{\headheight}{16pt}
\lhead{Lukas Hempel \& Thomas Pause}
\chead{Abgabe Aufgabe 2}
\rhead{\today}%{Matrikelnummer: 3720245}
\cfoot{\thepage}
\setlength{\parindent}{0pt}
%\addbibresource{bibliography.bib}

\newcommand{\primary}[1]{$\uline{\texttt{#1}}$}
\newcommand{\foreign}[1]{$\dashuline{\smash{\texttt{#1}}}$}
\newcommand{\both}[1]{$\dashuline{\uline{\texttt{#1}}}$}

%for sql syntax
\usepackage{listings}
\lstset{
    numberbychapter=false,
    numbers=none,
    float=htb,
    numberblanklines=true,%specified even when numbers=none here for the case that numbers are explicitly activated for a single listing
    %numbers=left,
    %numberstyle=\tiny,
    %basicstyle=\ttfamily\fontsize{11}{11}\selectfont,
    basicstyle=\normalfont\ttfamily,
    tabsize=2,
    framexleftmargin=2pt,
    captionpos=b,
    %frame=single,
    breaklines=true,
    %keywordstyle=\color{cyan}, % screen version
    keywordstyle=\color{blue}, % print version
    %stringstyle=\color{listkeyword}, % print version % does not seem to work correctly
    aboveskip=1.5em
}
\lstset{language=sql,breaklines=true}
\lstdefinelanguage{sql}{
sensitive=true,
morekeywords={TABLE, ON, UPDATE, DELETE, CREATE, OPERATOR, FUNCTION, SELECT,DISTINCT, FROM, WHERE, FILTER, ORDER, GROUP, BY, HAVING, IN, AS, CONSTRAINT, GRAPH, SERVICE, PREFIX, ASC, DESC, LIMIT, SUM, MIN, MAX}
}
\lstset{basicstyle=\ttfamily}
\lstset{literate=%
  {Ö}{{\"O}}1
  {Ä}{{\"A}}1
  {Ü}{{\"U}}1
  {ß}{{\ss}}1
  {ü}{{\"u}}1
  {ä}{{\"a}}1
  {ö}{{\"o}}1
} % für Umlaute in lstlistings
\lstset{language=Python,upquote=true}

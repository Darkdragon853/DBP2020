\documentclass[11pt,titlepage]{article}
\usepackage[utf8]{inputenc}
\usepackage{amsmath}
\usepackage{amssymb}
\usepackage[ngerman]{babel}
\usepackage{graphicx}
\usepackage{comment}
\usepackage{appendix}
\usepackage{caption, subcaption}
\usepackage[T1]{fontenc}
\usepackage{multirow}

\usepackage{pdfpages}
% für Unterstreichungen
\usepackage[normalem]{ulem}
%\usepackage{environ}
\usepackage{fancyhdr}
% \usepackage[
%   sorting=none
%   %backend=biber,
%   %style=numeric,
% ]{biblatex}

\usepackage[
  a4paper,
  includeheadfoot,
  left=20mm,
  right=20mm,
  top=20mm,
  bottom=20mm
]{geometry}

\usepackage{csquotes}
\usepackage{wrapfig}
\usepackage{longtable}
\usepackage{siunitx}

\usepackage{tabularx}
\usepackage{hyperref}
\usepackage{setspace}

\pagestyle{fancy}
\setlength{\headheight}{16pt}
\lhead{Lukas Hempel \& Thomas Pause}
\chead{Abgabe Aufgabe 1}
\rhead{\today}%{Matrikelnummer: 3720245}
\cfoot{\thepage}
\setlength{\parindent}{0pt}
%\addbibresource{bibliography.bib}

\newcommand{\primary}[1]{$\uline{\texttt{#1}}$}
\newcommand{\foreign}[1]{$\dashuline{\smash{\texttt{#1}}}$}
\newcommand{\both}[1]{$\dashuline{\uline{\texttt{#1}}}$}

%for sql syntax
\usepackage{listings}
\lstset{
    numberbychapter=false,
    numbers=none,
    float=htb,
    numberblanklines=true,%specified even when numbers=none here for the case that numbers are explicitly activated for a single listing
    %numbers=left,
    %numberstyle=\tiny,
    %basicstyle=\ttfamily\fontsize{11}{11}\selectfont,
    basicstyle=\normalfont\ttfamily,
    tabsize=2,
    framexleftmargin=2pt,
    captionpos=b,
    frame=single,
    breaklines=true,
    %keywordstyle=\color{cyan}, % screen version
    keywordstyle=\color{blue}, % print version
    %stringstyle=\color{listkeyword}, % print version % does not seem to work correctly
    aboveskip=1.5em
}
\lstset{language=sql,breaklines=true}
\lstdefinelanguage{sql}{
sensitive=true,
morekeywords={create, table, select, SELECT,DISTINCT,SAMPLE, FROM ,WHERE,FILTER,ORDER, GROUP, BY,HAVING,IN,AS,GRAPH,SERVICE,PREFIX,ASC,DESC,LIMIT,SUM,MIN,MAX}
}


\begin{document}

\begin{titlepage}

    \begin{center}

        %\includegraphics[width=0.8\textwidth]{img/header.png}\\
        \vspace{\fill}
        \huge{Datenbankpraktikum 2020}\\

        \vspace{20pt}

        \hrule
        \vspace{26pt}
        \Huge \textbf{Abgabe Aufgabe 1}
        \vspace{20pt}
        \hrule

        \vspace{10pt}

        \vspace{\fill}
        \Large{\textbf{Lukas Hempel~\&~Thomas Pause}} %~~ \Large{Matrikelnummer: 3720245}
        ~\\
        \Large{Matrikelnummern: 0000000~\&~3720245}

        \Large{Bachelor Informatik}
        ~\\
        \Large{Betreuer: bla}
        \vspace{\fill}

    \end{center}

    \begin{flushright}

        \Large{\textbf{Termin 1.Testat:}} ~\Large{28.05.2020}
    \end{flushright}

\end{titlepage}


\section*{2a) Sichtenerstellung}
Die Freundschaftsbeziehung ist als gerichtete Beziehung gespeichert, um Anfragen bzgl. der Freundschaftsbeziehung komfortabel zu lösen sollen die Beziehungen ungerichtet gespeichert werden.
Diesbezüglich sollen Sie eine Sicht \enquote{pkp\_symmetric} erstellen, die beide Richtungen enthält.

\section*{2b) Anfragen an die Datenbank}
Formulieren Sie SQL-Anfragen, um folgende Fragen zu beantworten.

\begin{enumerate}
    %\item In wie vielen verschiedenen afrikanischen Städten gibt es eine Universität?
    \item Wie viele Forenbeiträge (Posts) hat die älteste Person verfasst (Ausgabe: Name, \#Forenbeiträge)?

    \uline{Query:}\\
    SELECT Name, Count(Post) AS \#Forenbeiträge \\
    FROM bla JOIN blubb \\
    WHERE namenszuordnung AND MAX(age)\\
    \uline{Ausgabe:}\\

    \item Wie viele Kommentare zu Posts gibt es aus jedem Land (Ausgabe aufsteigend sortiert nach Kommentaranzahl)? Die Liste soll auch Länder enthalten, für die keine Post-Kommentare existieren, d.h. die Kommentaranzahl = 0 ist! (Funktion Coalesce)

    \uline{Query:}\\
    SELECT ...\\
    \uline{Ausgabe:}\\

    \item Aus welchen Städten stammen die meisten Nutzer (Ausgabe Name + Einwohnerzahl)?

    \uline{Query:}\\
    SELECT ...\\
    \uline{Ausgabe:}\\

    \item Mit wem ist \enquote{Hans Johansson} befreundet?

    \uline{Query:}\\
    SELECT ...\\
    \uline{Ausgabe:}\\

    \item Wer sind die \enquote{echten} Freundesfreunde von \enquote{Hans Johansson}? \enquote{Echte} Freundesfreunde dürfen nicht gleichzeitig direkte Freunde von \enquote{Hans Johansson} sein. Sortieren Sie die Ausgabe alphabetisch nach dem Nachnamen.

    \uline{Query:}\\
    SELECT ...\\
    \uline{Ausgabe:}\\

    \item Welche Nutzer sind Mitglied in allen Foren, in denen auch \enquote{Mehmet Koksal} Mitglied ist (Angabe Name)?

    \uline{Query:}\\
    SELECT ...\\
    \uline{Ausgabe:}\\

    \item Geben Sie die prozentuale Verteilung der Nutzer bzgl. ihrer Herkunft aus verschiedenen Kontinenten an!

    \uline{Query:}\\
    SELECT ...\\
    \uline{Ausgabe:}\\

    \item Zu welchen Themen (\texttt{tag classes}) gibt es die meisten Posts? Geben Sie die Namen der Top 10 \texttt{tag classes} mit ihrer Häufigkeit aus!

    \uline{Query:}\\
    SELECT ...\\
    \uline{Ausgabe:}\\

    \item Welche Personen haben noch nie ein \enquote{Like} für einen Kommentar oder Post bekommen? Sortieren Sie die Ausgabe alphabetisch nach dem Nachnamen.

    \uline{Query:}\\
    SELECT ...\\
    \uline{Ausgabe:}\\

    \item Welche Foren enthalten mehr Posts als die durchschnittliche Anzahl von Posts in Foren (Ausgabe alphabetisch sortiert nach Forumtitel)?

    \uline{Query:}\\
    SELECT ...\\
    \uline{Ausgabe:}\\

    \item Welche Personen sind mit der Person befreundet, die die meisten Likes auf einen Post bekommen hat? Sortieren Sie die
    Ausgabe alphabetisch nach dem Nachnamen.

    \uline{Query:}\\
    SELECT ...\\
    \uline{Ausgabe:}\\

    \item Welche Personen sind direkt oder indirekt mit \enquote{Jun Hu} (id 94) verbunden (befreundet)? Geben Sie für jede Person die Distanz zu Jun an.

    \uline{Query:}\\
    SELECT ...\\
    \uline{Ausgabe:}\\

    \item Erweitern Sie die Anfrage zu Aufgabe 13 indem Sie zusätzlich zur Distanz den Pfad zwischen den Nutzern ausgeben.

    \uline{Query:}\\
    SELECT ...\\
    \uline{Ausgabe:}\\

\end{enumerate}

\section*{2c) Änderungen in der erzeugten Datenbank}
Es soll ein Mechanismus umgesetzt werden, um die Beendigung eines Arbeitsverhältnisses zu dokumentieren.
Der entsprechende Eintrag in \texttt{person\_workAt\_company} soll mittels SQL-Anweisung gelöscht werden.
Um die Datenmanipulation nachvollziehen zu können, soll der Löschvorgang in einer separaten Tabelle protokolliert werden.
Dabei soll zusätzlich hinterlegt werden, wann das Arbeitsverhältnis beendet wurde (orientieren Sie sich am Löschzeitpunkt).
Die Protokollierung soll automatisch erfolgen, wenn ein Mitarbeiter sein Arbeitsverhältnis bei einem Unternehmen beendet (Löschung in \texttt{person\_workAt\_company}).

\end{document}
